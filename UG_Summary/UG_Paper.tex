\documentclass[a4paper,10pt]{article}
\usepackage[utf8]{inputenc}
\usepackage[linguistics]{forest}
\usepackage[round,sort,comma,authoryear]{natbib}

\newcommand{\mli}[1]{\mathit{#1}}

\begin{document}

\begingroup% Robert Frost, T&H p 149
\centering
\vfill
\Huge{Noam Chomsky's}\\[0.5\baselineskip]
\Huge {UNIVERSAL GRAMMAR}\\
\Huge{\&}\\
\Huge{The Current State of The Theory}\\[\baselineskip]
\Large {Samuel Pearce}\par
\large{\scshape 2021}\par
\vfill\null
\endgroup

\begin{abstract}
    Noam Chomsky's Theory of Universal Grammar has been the defining
    theory behind linguistics for the past several decades. Many
    linguists have based their career around how they stand in relation
    to it, and no matter one's stance on the matter, it is impossible
    to refute the Theory's monumental impact on the field as a whole.
    This paper aims to outline --- very briefly --- what the theory
    entails, how it has evolved since it was first proposed in the
    1960s, and what the most current consensus on the matter is.
\end{abstract}

\pagebreak


\tableofcontents
\pagebreak


\section{Universal Grammar}
To begin with, what is universal grammar exactly, and what does it entail? Well, contrary to what
I originally thought, it is not a defined linguistic grammar that applies to all human languages.
The theory of universal grammar mainly consists of the hypothesis that language is an innate human
faculty that is defined --- at least, to some extent --- by our biology and genetics. One of the
well-known arguments Chomsky presents is that young children can understand grammatical concepts
about the language they speak without having been taught it directly.\citep[p.~26]{ChUGAI}

At its base, UG-theory suggests there is some kind of ``computation system'' that converts the
external signals (sounds, symbols, sign-language, etc.) into internal ideas. \citep[p.~5]{ChUGAI}
Most of what Chomsky worked on after the theory was put into place, was creating a formal system
to analyse syntax of varying languages to compare which elements could be universal. The ideas
and structures used to analyse language and form ideas about how such an internal system might
be structure has been discussed and evolved over the past nearly half century:
\begin{quote}
	``The UG Theory claims to be a scientific theory based on solid evidence about language.
	As such, it is always progressing towards better explanations for language knowledge....''
	
	\citep[p.~27]{ChUGAI}
\end{quote}

The theory split into two main branches, each focusing on one end of the computational system;
they were the ``E-Language'' and ``I-Language'' branches. E-Language is primarily concerned with
the actual physical manifestation of language by gathering large samples and analysing them. The
I-Linguists focus on mapping out how these ideas are stored in the brain, whith the details of
how they are expressed mostly irrelevant. To summarise: E-Language ``is concerned with what people
have done'', while I-Language ``is concerned with what they could do'' \citep[p.~14]{ChUGAI}

One of the first things one would think to analyse when speaking about a ``universal'' grammar
are so-called linguistic universals; features that have the same structure throughout all human
languages. Greenbergian universals, linguistic features like syntax structure or movement rules
that appear in all natural languages \citep{Greenberg473}, differ from Chomskyan universals in that
Chomskyan universals needn't manifest in every language; ``No language violates a universal principle
(the language simply may not use the principle in a particular context)'' \citep[p.~23]{ChUGAI}


\section{Generative Grammar}
A generative grammar is a description of a language that uses a very explicit syntactic syntax.
Chomsky himself defined it thusly:

\begin{quote}
	``When we speak of the linguist's grammar as a `generative grammar' we mean only that it is
	sufficiently explicit to determine how sentences of the language are in fact characterised
	by the grammar''
	
	\citep[p.~220]{Chomsky80}
\end{quote}

It is based on the idea of building up sentence syntax the same way one can with
a mathematical grammar, like a programming language, possible syntax trees are written as recursive
rewrite rules such as:

\[        S \rightarrow \mli{NP} \; \mli{VP} \]
\[ \mli{VP} \rightarrow V        \; \mli{NP} \]
\[ \mli{NP} \rightarrow Det      \; N        \]

In the above example a sentence (S) is defined as a noun-phrase (NP) plus a verb-phrase (VP),
a verb-phrase consists of a verb (V) and a noun-phrase, and a noun-phrase is a determiner (Det) plus
a noun (N). With these rules you can construct some of the many possible grammatically valid sentences
for English. \citep[p.~32]{ChUGAI}

Though this was just the beginning of a new way of analysing the syntactic structures of sentences
which lead to many further developments.


\section{Government/Binding Model}
In the new government/binding model, the simple building blocks of language which were still being
defined were stuck together modularly to create a model of how the computational system could be
structured. At its base, the structure has two layers: the D-structure (deep structure) and the
S-structure (surface structure). The D-structure reflects the base grammatical structure of the
sentence, influenced by the lexicon and phrase structure rules explained above, while the S-structure
is what is actually said after being adjusted by movement rules. \citep[p.~61]{ChUGAI}
For example, in the following sentence, the ``whom'' would be the object of the sentence --- coming
after the verb as usual --- in the D-Structure, but because it's a question, the object is moved
to the front of the sentence:

\begin{center}
	* You did see whom. $\rightarrow$ Whom did you see?
\end{center}

This was the basis of the GB theory which would continue to be developed with new additions to
help handle all cases seen in human languages that didn't already meet the basic model.


\section{X-Bar Theory}
In the earlier example we stated that a verb-phrase was defined as a verb and a noun-phrase, but
some verbs are intransitive and don't take an object. While we could simply define two possible
verb-phrase trees --- one with a object and one without --- this information is based on the word
itself and this causes an unnecessary redundancy. Chomsky's rewrite rules were then expanded upon
with the introduction of X-bar notation to remove this redundancy; a phrase could be defined using
only following rewrite rules:

\[ X'  \rightarrow  \mli{X}   \; (\mli{YP})\footnotemark[1] \]
\[ X'' \rightarrow (\mli{YP})\footnotemark[2] \;  \mli{X'}  \]

\footnotetext[1]{The complement is optional, it is determined by the head}
\footnotetext[2]{The specifier is also optional, but isn't selected by the head}

\begin{center}
\begin{forest}
[X$^{\prime\prime}$
	[specifier]
	[X$^{\prime}$
		[X]
		[complement]
	]
]
\end{forest}
	\linebreak
	Fig. 1: The X-Bar rules displayed in a tree
\end{center}

Substitute X and Y with any of the four lexical categories in the theory
--- Noun (N), Verb (V), Adjective (A), Preposition (P) --- and you have the structure of that phrase.
This system can represent any phrase possible by recursively adding more elements to the tree.
\citep[p.~66]{ChUGAI}

\begin{center}
\begin{forest}
[S
	[N$^{\prime\prime}$ [he, roof]]
	[V$^{\prime\prime}$
		[V$^{\prime}$
			[V [flew]]
			[P$^{\prime\prime}$
				[P$^{\prime}$
					[P [to]]
					[N$^{\prime\prime}$ [London, roof]]
				]
			]
		]
	]
]
\end{forest}
	\linebreak
	Fig. 2: An example of recursive phrases from \citet[p.~68]{ChUGAI}
\end{center}


\section{Principles \& Parameters}


\section{Minimalist Program}


\section{Criticism}


\section{My Impression}
(Should I include this?)


\pagebreak
\bibliographystyle{plainnat}
\bibliography{Bibliography}

\end{document}
