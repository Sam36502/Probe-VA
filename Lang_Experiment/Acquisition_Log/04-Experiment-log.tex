\documentclass[a4paper,10pt]{article}
\usepackage[utf8]{inputenc}

\usepackage{leipzig}
\usepackage{gb4e}

%opening
\title{Language Acquisition Experiment Log}
\author{Samuel Pearce}

\begin{document}

\maketitle

\tableofcontents

\pagebreak

\section{Experiment Parameters}

\begin{itemize}
 \item Duration:        From June 1st to June 25th, 2021.
 \item Experimenter:    Samuel Pearce
 \item Participant A:   Julian Werner, Male, 18
 \item Participant B:   Amin Haidar Male, 19
\end{itemize}


\subsection{Experiment Rules For Participant B}
\begin{enumerate}
 \item The experimenter is, at no point during the experiment period,
 allowed to explain anything about Toki Pona to the participant in
 any language other than Toki Pona.
 \item The participant is not allowed to independently seek resources
 about Toki Pona. Any external resources for them are to be provided by
 the experimenter.
 \item The participant may take notes and review them outside of
 experimentation times.
 \item The experimenter may not confirm or deny
 any of the participant's notes or guesses.
\end{enumerate}

\subsection{Notes on glosses}
A few non-standard gloss abbreviations and elements are used
throughout this document, so I have defined their meaning here:
\begin{itemize}
 \item The pictographic
 elements will be replaced in the glosses with simple descriptions of their
 contents and meaning in square brackets ([ Image ; Intended Meaning ])
 E.g.: [Tree Image; Plants/Trees/Nature]
 \item Toki Pona includes context phrases that add extra information
 about the context of a sentence and build conditionals. The separator
 will be abbreviated as ``CTXT'' in glosses.
\end{itemize}



\section{Session 1 - June 1st}
The first session began in a text chat with the ability to send emoji
and images. The experimenter attempted to establish the basic sentence
structure with comprehensible pictographic substitution.

\begin{exe}
 \ex
 \gll pona li [Thumbs Up; Good] \\
 good PM [Thumbs Up; Good] \\
 \glt ``pona'' (is) ``good''
\end{exe}
\begin{exe}
 \ex
 \gll ike li [Thumbs Down; Bad] \\
 bad PM [Thumbs Down; Bad] \\
 \glt ``ike'' (is) ``bad''
\end{exe}

Which was generally perceived correctly, but the participant
did not understood the meaning of the subject-verb separator ``li''.
The next concepts to be introduced were ``yes/correct'' and ``no/incorrect''
to enforce the participant's correct usage of phrases.

\begin{exe}
 \ex
 \gll ala li [Red X-Shape; No/Incorrect] \\
 no PM [Red X-Shape; No/Incorrect] \\
 \glt ``ala'' (is) ``No/Incorrect''
\end{exe}
\begin{exe}
 \ex
 \gll lon li [Green Checkmark; Yes/Correct] \\
 yes PM [Green Checkmark; Yes/Correct] \\
 \glt ``lon'' is ``Yes/Correct''
\end{exe}

After the basic structure was grasped, the word for person --- a prefix
required to write names in Toki Pona --- was explained along with the
basic pronouns by using the participant's names. For this, it was
fortunate that there were at least three people present, in order to
explain the concepts of first, second, and third person.

\begin{exe}
 \ex
 \gll mi jan. sina jan. [Face; Person] li jan \\
 1SG be-person. 2SG be-person. [Face; Person] PM be-person \\
 \glt I am a person. You are a person. People are people.
\end{exe}

At this explanation, the participant independently responded with variations
on the following:

\begin{exe}
 \ex
 \gll jan li [Face; Person] \\
 person PM [Face; Person] \\
 \glt ``jan'' (is) ``person''
\end{exe}

To which, the experimenter could now respond with ``lon'' meaning, ``Yes'' or
``Correct''. Given the informal nature of the experiment, the participant
took the opportunity of having the basic ``X is Y'' sentence structure
to insult the other participant and the experimenter, in a jovial fashion,
by saying variations of the following:

\begin{exe}
 \ex
 \gll jan Julian li ike. \\
 person Julian PM be-bad \\
 \glt Julian is bad.
\end{exe}

It was around this time that correct Toki Pona spelling was explained, through
emphasized corrections. The participant used the other's name in lower case,
which the experimenter corrected by responding to it, with the corrected capital
letter emphasized. This concept was quite quickly understood.

Once everyone began speaking face-to-face, the experimenter explained the Toki Pona
demonstrative ``ni'' by pointing at various things and explaining what they
were using the established vocabulary. 

\begin{exe}
 \ex
 \gll ni li ijo. \\
 this PM be-thing \\
 \glt This is a thing.
\end{exe}

With the basic framework of these 10 words, explaining other novel concepts
through the use of images was quite easy and got both participants forming
basic sentences to describe their environment and what they were doing
quite quickly:

\begin{tabular}{ r | l }
 Toki Pona & English \\
 \hline
 li & Predicate Marker \\
 mi & I, me, my, we, us, our \\
 sina & you, your \\
 ona & he, she, it, him, her, his, hers, its \\
 pona & good, simple \\
 ike & bad, complicated \\
 lon & yes, correct, true \\
 ala & no, not, none \\
 ni & this, that \\
 ijo & thing \\
 \hline  
\end{tabular}

\hfill\break
After some basic nouns were learned through pictograms, the more complicated
concept of motion and giving/receiving things was discussed. At first the
word ``tawa'', meaning ``to move'' was explained by the experimenter getting
up and walking across the room while saying ``mi tawa.'' (``I go/walk.'').
After this, the preposition form of ``tawa'' was explained using the help
of the word ``pana'' which means ``to give''. The participant handed
the experimenter a book and asked what they was doing, to which the experimenter
replied:

\begin{exe}
 \ex
 \gll sina pana e ni tawa mi. \\
 2SG give ACC this PREP.to 1SG \\
 \glt You are giving this to me.
\end{exe}

After a few clarifications and re-enunciations, the participant understood
the meaning. At around this point, They began compiling a  dictionary.

The final concept explained
was the opposite of ``tawa'': that of ``kama'' meaning ``to arrive/come'' or ``to become''
and the concept of ``tan'' meaning ``from''. ``kama'' was fairly easily
explained by a pictograph of a person walking in the opposite direction
from a similar pictograph labelled ``tawa''. After that, the experimenter
attempted to strengthen the understanding by passing around the book again
and trying to explain that the book came from him:

\begin{exe}
 \ex
 \gll mi pana e ni tawa sina. ni li kama tan mi. \\
 1SG give ACC this PREP.to 2SG. this PM come PREP.from 1SG\\
 \glt I give this to you. This came from me.
\end{exe}

Though the efforts were not fruitful, it eventually occurred to the experimenter
to use the mutual knowledge of each other's lives to cement tan's meaning:

\begin{exe}
 \ex
 \gll mi kama tan ma New Zealand\footnotemark. \\
 1SG come PREP.from country New Zealand.\\
 \glt I come from New Zealand.
\end{exe}
\footnotetext{New Zealand would normally be referred to as ``ma Nusilan''--- like all
foreign words --- to conform to Toki Pona's phonetics, but it was decided to choose the
style of simply using words' native forms, to simplify referring to external subjects.}

This was immediately understood and served to make the participants sure of
their guesses as to its meaning.

This first session ended and all members discussed (in English) the plans
of when to meet and what the scope of the experiment should be. It was suggested that they could
also learn Sitelen Pona, Toki Pona's own logographic writing system. This
was left open as an option depending on how much the participants were able
to learn in the time allotted. Both participants seem enthusiastic to
continue the experiment out of their own curiosity.


\section{Session 2 - June 6th}

To begin this session, the next most important beginner phrase was
explained: ``What is this?''. By explaining this, the participant would
be able to request new information or ask for clarification of existing
information independently. A dialogue was used to clear up the meaning:

\begin{exe}
 \ex
 \gll A: ni li seme? B: ni li moku. A: a! mi sona! \\
 A: this PM be-what? B: this PM be-food. A: EXCL! 1SG understand! \\
 \glt A: What is this? B: This is food. A: Ah! I understand!
\end{exe}

Next, the basic colours were explained to participant B
through pictures. The following phrases were told which they immediately
understood.

\begin{exe}
 \ex
 \gll [Image of colour spectrum; Colour] li `kule'. \\
 [Image of colour spectrum; Colour] PM be-colour.\\
 \glt Colour is ``kule''.
\end{exe}
\begin{exe}
 \ex
 \gll [Red Circle; Red] li `loje'. \\
 [Red Circle; Red] PM be-red.\\
 \glt Red is ``loje''.
\end{exe}
\begin{exe}
 \ex
 \gll [Green and Blue Circle; Grue] li `laso'. \\
 [Green and Blue Circle; Grue] PM be-grue.\\
 \glt Grue (Green \& Blue) is ``laso''.
\end{exe}
\begin{exe}
 \ex
 \gll [Yellow Circle; Yellow] li `jelo'. \\
 [Yellow Circle; Yellow] PM be-yellow.\\
 \glt Yellow is ``jelo''.
\end{exe}

The participant's knowledge was then confirmed by asking the following:

\begin{exe}
 \ex
 \gll kasi li jo e kule seme? \\
 plants PM have ACC colour what?\\
 \glt What colour do plants have?
\end{exe}

Which was answered with ``laso''. The correct answer was reinforced
with a ``lon'' and a few more similar examples were tested.
After the session, the participants were polled on the idea of
learning the writing system, which they were against given the
very tight time scope of the project.

\section{Session 3 - June 4th}
During an impromptu gaming session, it was suggested that participant B
and I attempt to play the video game Minecraft while only conversing
in Toki Pona. This served mainly to expand the vocabulary, as the most
basic required grammatical concepts were already fairly well understood.

The virtual environment provided an excellent way to strengthen connections
through visual confirmation; the experimenter could point to an in-game item
and explain its meaning with the participant able to respond in kind. At
this point vocabulary like ``Stone'', ``Wood'', ``Table'', ``House'' and
``Sleep'' could be acquired through direct examples.

Near the end of the session, the concept of time and tenses was attempted
to be explained by use of the in-game day/night cycle. The word for sky
was explained by stating ``The sky is blue.'' during the day and ``The
sky is black.'' during the night. Next, names for day and night and the
heavenly bodies were explained by pointing them out when they were visible.
``tenpo pimeja'' was soon picked up to mean ``night time'' (LIT. ``dark time'').
Finally, tenses were explained with the following phrases after a night of
sleep:

\begin{exe}
 \ex
 \gll tenpo pini, la sewi li pimija. \\
 time finished, CTX sky PM be-dark.\\
 \glt LIT. In finished time, the sky is dark.
 \glt ``Then, the sky was dark.''
\end{exe}

\begin{exe}
 \ex
 \gll tenpo ni, la sewi li walo. \\
 time this, CTX sky PM be-light.\\
 \glt LIT. In this time, the sky is bright.
 \glt ``Now, the sky is bright.''
\end{exe}

\begin{exe}
 \ex
 \gll tenpo kama, la sewi li pimija. \\
 time coming, CTX sky PM be-dark.\\
 \glt LIT. In coming time, the sky is dark.
 \glt ``Soon, the sky will be dark.''
\end{exe}

It took a few examples and laying out the tenses on a
physical timeline, but the participant eventually
understood.

\end{document}
