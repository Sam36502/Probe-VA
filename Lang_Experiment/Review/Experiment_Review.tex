\documentclass[a4paper,10pt]{article}
\usepackage[utf8]{inputenc}
\usepackage[linguistics]{forest}
\usepackage[round,sort,comma,authoryear]{natbib}

\newcommand{\mli}[1]{\mathit{#1}}

\begin{document}

\begingroup% Robert Frost, T&H p 149
\centering
\vfill
\Huge{Review}\\
\Large{of an}\\
\Huge {EXPERIMENT}\\
\Large{in}\\
\Huge{LANGUAGE ACQUISITION}\\[\baselineskip]
\Large {Samuel Pearce}\par
\large{\scshape 2021}\par
\vfill\null
\endgroup

\begin{abstract}
	To test some personal theories on language acquisition vs. language learning, I undertook to
	teach two of my friends Toki Pona\footnotemark over the course of my probe-VA. The goal was to
	gain a better understanding of the process of teaching someone a language, what difficulties arise
	when teaching purely through immersion, and what the most prominent differences between two methods
	of language learning: immersion and the traditional approach of vocabulary and grammar books, as
	well as speculating on the optimal method of language learning. As one might expect, the answer is
	not as simple as just one or the other.
\end{abstract}

\footnotetext{
	Toki Pona is a minimalist language constructed by Sonja Lang in 2001. It has a very minimal
	lexicon which lends itself well to such a short experimentation period. I also am already
	fluent in it, which made it easy to immerse the participant.
}

\pagebreak


\tableofcontents
\pagebreak


\section{Experiment Parameters}
The experiment consisted of teaching both participants the language Toki Pona. Participant A would
learn the language the ``traditional'' way: using vocabulary and grammar textbooks, while participant
B would be forbidden from having any words or concepts explained directly through a third language;
everything would have to be learnt through gesture, pictures, and explanations using previously
explained words. Once the roughly three weeks were over, I would record the participants and I having
a conversation in order to analyse the participants' fluency. Additionally, a test would be performed
where I would give one participant some specific thing to describe and the other would have to explain
to me what they were talking about to evaluate their explanation and comprehension skills more directly.


\section{Hypothesis}
My understanding of the language learning and language teaching process at the beginning of the
experiment is quite limited. Until now, I have only ever tried teaching a language to my
parents, which did not go very well. My expectations for how the process will work is that
I will spend fixed periods of time with each participant explaining various concepts and being
available for questioning as often as possible.

As for the taching itself, I guess that conveying the extremely vague Toki Pona words' meanings
through immersion alone without confusing the participant will be quite difficult and could lead to
many issues in comprehension. Though, I assume the amount of practice afforded to the immersion
participant would drastically improve their immediate comprehension and speaking ability. My idea
for the traditional-learning method would be that the participant will very quickly pick up a
larger amount of the lexicon with more concrete understanding than the immersion participant, but
will struggle to recall and apply the grammar to them rapidly enough for speech.

As for how the end result would be, my personal hypothesis on the participants' grasp on the language
would be that, in general, the immersion participant would be able to recall the most common vocabulary
and how to properly implement them into the grammar far quicker than the traditional-learning
participant; however, I guessed that the traditional-learning participant would have a more correct
usage of the language according to its official grammar than the immersion participant, while also
being able to convey more complex topics due to their superior faculty over the vocabulary, but they
would lack the speed required to utilise this knowledge in speech.

These are my hypotheses regarding how the experiment will go.


\section{Pure Immersion Method}
For both methods, the expectation that I would be able to spend regular sessions with the participants
was quickly disregarded as unrealistic. Instead, due to the Coronavirus pandemic and our individual
schedules, it made the most sense to sporadically practice the language through mostly digital
interactions.

For the immersion participant, this entailed either discussing topics through a text-channel, which
helpfully provided to opportunity to explain basic concepts through the use of pictographic imagery,
and meeting --- either in-person, or in a video game --- the latter of which provided both ample
example material and a basic system of gesturing which helped convey the meanings of my utterances.

In total, I would estimate that we spent about 6 hours speaking the language. The first few sessions
were very difficult, as I had far fewer tools at my disposal to explain the concepts I was trying to
convey, but once the most basic grammatical features were understood, I was able to bootstrap the
participant's understanding by progressively building on top of the previously understood material.

// INSERT EXAMPLE GLOSSES FROM LOG IN RANDOM PLACES HERE

Eventually, the most important grammar rules were understood and the bulk of the remaining words to
be taught were simple verbs which could be explained through a picture accompanied by the phrase
``This is X''.


\section{Vocab \& Grammar Method}
The traditional-learning participant proved far easier to educate, simply because they were able to
educate themselves with very little supervison; I gave them a copy of the official Toki Pona grammar
book, which includes lessons about all the grammar as well as a dictionary and some sample texts.

I periodically checked in with the progress he was making, which was very sluggish at first, but with
a deadline looming, they began by learning the entire lexicon in one go, given it's small size, which
they stated improved their ability to understand the grammar through the examples, now that they knew
what all the words meant.

// INSERT Q/A example

This participant estimates they spent /// INSERT HOUURSE HERE /// hours studying the language
independently. It really showed off one of the major benefits of this method of learning: it's
completely possible to do without a teacher and allows the student to proceed at their own pace.
The immersion participant lacked this independence, and even though the lessons were tailored to fit
their pace of learning, this was only possible, because I only had one student. If you tried to teach
any significant quantity of people the language, people would likely be forgotten.


\section{Results}

/// INSERT TRANSCRIPT OF DIALOGUE / CONVEYANCE TEST THING ///

\section{Review}
My expectations for how teaching would be, were quite drastically adjusted after I had to try explaining
even basic concepts such as thanking someone through the bare minimum language available, and the
simple convenience of independent study was something that suprised me. Though, I still believe the
best option is a mix of independence, a structured lesson-plan, and a very large amount of input.

The thing that suprised me most about teaching through immersion for the first time was how intuitive
it was. There were certainly many frustrating moments where I wished I could just say what the word
was in English, but we always managed to find a way to come to an understanding, even if we sometimes
had to rely on pre-existing knowledge about one-another and our culture to convey it. Ideally, the
experiment would be conducted on a large number of carefully vetted participants, but one must work with
the materials available.

/// INSERT RESULT-RELEVANT COMMENTS ABOUT EACH PARTICIPANT'S FACULTY ///


\end{document}
